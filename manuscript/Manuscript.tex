\documentclass[nonblindrev]{write_paper} 
\usepackage{amsmath}
\usepackage{ctex}
\usepackage{url}
%\usepackage{algorithm}
\usepackage{booktabs}
\usepackage{multirow}
\usepackage{algpseudocode}
\usepackage{subfigure}
\usepackage{epsfig}
\usepackage[utf8]{inputenc}
\usepackage[linesnumbered, ruled, vlined]{algorithm2e}
\usepackage{amsmath}
%链接点击跳转
\usepackage[hidelinks]{hyperref}
\hypersetup{
  colorlinks=true,
  linkcolor=blue,
  filecolor=magenta,      
  urlcolor=blue,
  citecolor=blue,
}
\makeatletter
\@ifundefined{newblock}{\def\newblock{\hskip .11em plus .33em minus .07em}}{}
\makeatother

\usepackage{enumitem}
\makeatletter
\renewcommand\section{\@startsection {section}{1}{\z@}%
                                   {-1ex \@plus -1ex \@minus -.1ex}%
                                   {1 ex \@plus.1ex}%
                                   {\normalfont\large\bfseries}}
\renewcommand\subsection{\@startsection{subsection}{2}{\z@}%
                                     {-1ex\@plus -1ex \@minus -.1ex}%
                                     {1ex \@plus .1ex}%
                                     {\normalfont \normalsize \bfseries}}
\renewcommand\subsubsection{\@startsection{subsubsection}{3}{\z@}%
                                     {-1ex\@plus -1ex \@minus -.1ex}%
                                     {1ex \@plus .1ex}%
                                     {\normalfont\normalsize\bfseries}}
\makeatother

\renewcommand{\theARTICLETOP}{}
\usepackage{fancyhdr}
\pagestyle{fancy}
\fancyhf{}
\fancyhead[LE,RO]{}
\fancyhead[RE,LO]{\scriptsize{计算金融与仿真课程论文} }
\fancyfoot[CE,CO]{\leftmark}
\cfoot{\thepage}

\usepackage[T1]{fontenc}
\usepackage{palatino}

\OneAndAHalfSpacedXI
\usepackage{color}
\usepackage{soul}

\DeclareMathOperator{\E}{\mathbb{E}}
\DeclareMathOperator{\R}{\mathbb{R}}
\DeclareMathOperator{\B}{\mathbb{B}}
\DeclareMathOperator{\Z}{\mathbb{Z}}

%%-----------------------------------------
%% 将作者-年份改为数字制
\usepackage[numbers]{natbib}  % 关键:numbers选项
% \bibpunct{[}{]}{,}{n}{}{,}   % 若需要可自行指定标点
%%-----------------------------------------

% 如果之前有 \bibpunct 设置, 请注释掉或删除以免冲突
% \bibpunct[, ]{(}{)}{,}{a}{}{,}%  <-- 原先作者-年份制, 需要注释或删除

\TheoremsNumberedThrough
\EquationsNumberedThrough
\MANUSCRIPTNO{} 
\newtheorem{prop}{{Proposition}}
%\newtheorem{lemma}{{Lemma}}

%\renewcommand{\algorithmicrequire}{{Input:}}
%\renewcommand{\algorithmicensure}{{Output:}}

\newtheorem{implication}{\noindent{Implication}}

\newcommand{\YH}[1]{{\color{blue}#1}}
\newcommand{\JJ}[1]{{\color{black}#1}}
\newcommand{\eat}[1]{}
\usepackage{graphicx}

\usepackage{endnotes}
\let\footnote=\endnote
\def\notesname{Endnotes}

\usepackage[symbol]{footmisc}
\renewcommand{\thefootnote}{\fnsymbol{footnote}}

\newcommand{\bs}[1]{\boldsymbol{#1}}
\newcommand{\ml}[1]{\mathcal{#1}}
\newcommand{\mb}[1]{\mathbb{#1}}

\begin{document}

% 整页垂直居中
\vspace*{\fill}

\begin{center}

    \includegraphics[width=0.4\textwidth]{Figures/校徽.png}\\[1cm]
       \Huge \bfseries
       \textbf{计算金融与仿真}\\[0.5cm]
       \Huge \bfseries
       \textbf{课程论文}\\[2cm]
    
    \Large
    \begin{table}[htbp]
    \centering
    \Large
    \begin{tabular}{ll}
    \textbf{论文题目 }: & 计算金融与仿真 \\
    \textbf{学生姓名}: & 李晶晶 \quad 张璐 \quad 朱冯婧 \\
    \textbf{指导老师}: & 邓志斌 
    \end{tabular}
    \end{table}

    


\end{center}

\vspace*{\fill}
\newpage





\section{投资组合选择介绍}
\section{模型建立}
\subsection*{决策变量}
\[
x_i \in \{0, 1\}, \quad i = 1, 2, \ldots, N
\]
其中,
\begin{itemize}
  \item $x_i = 1$ 表示选择资助第 $i$ 个贷款对象;
  \item $x_i = 0$ 表示不选择。
\end{itemize}

\subsection*{参数说明}
\begin{itemize}
  \item $A_i$:第 $i$ 个贷款的金额;
  \item $r_i$:第 $i$ 个贷款的利率;
  \item $P_i$:第 $i$ 个贷款的违约概率;
  \item $B$:总投资预算;
  \item $R_{\max}$:允许的最大违约风险;
  \item $G_k$:第 $k$ 个信用等级的贷款集合;
  \item $\alpha_k$:第 $k$ 个信用等级的最大投资比例;
  \item $m$:最多选择的贷款数量(Top-$m$)。
\end{itemize}

\subsection*{模型形式}
\begin{equation}
\begin{aligned}
\max_{x_i \in \{0,1\}} \quad & \sum_{i=1}^N x_i A_i \left[ r_i(1 - P_i) - P_i \right] \\
\text{s.t.} \quad
& \sum_{i=1}^N x_i A_i \le B \quad \text{(预算限制)} \\
& \sum_{i \in G_k} x_i A_i \le \alpha_k B, \quad \forall k \quad \text{(信用等级比例约束)} \\
& \sum_{i=1}^N x_i A_i P_i \le R_{\max} \quad \text{(风险控制)} \\
& \sum_{i=1}^N x_i \le m \quad \text{(Top-$m$ 选择)}
\end{aligned}
\label{eq:main_model} 
\end{equation}


\begin{equation}
\begin{aligned}
\max_{x_i \in \{0,1\}} \quad & \sum_{i=1}^N x_i A_i \left[ r_i(1 - P_i) - P_i \right] \\
\text{s.t.} \quad
& \sum_{i=1}^N x_i A_i \le B \quad \text{(预算限制)} \\
& \sum_{i \in G_k} x_i A_i \le \alpha_k B, \quad \forall k \quad \text{(信用等级比例约束)} \\
& \sum_{i=1}^N x_i A_i P_i \le R_{\max} \quad \text{(期望风险控制)} \\
& \sum_{i=1}^N x_i \le m \quad \text{(Top-$m$ 选择)} \\
& \sum_{i=1}^N x_i A_i \cdot \tilde{P}_i^{(s)} - \eta \le \mathcal{M} z_s, \quad \forall s=1,\dots,S \quad \text{(VaR限制)} \\
& \sum_{s=1}^S z_s \le (1 - \beta) S \\
& z_s \in \{0,1\}, \quad \eta \in \mathbb{R}
\end{aligned}
\label{eq:var_model}
\end{equation}
\begin{equation}
\begin{aligned}
\max_{x_i \in \{0,1\}} \quad & \sum_{i=1}^N x_i A_i \left[ r_i(1 - P_i) - P_i \right] \\
\text{s.t.} \quad
& \sum_{i=1}^N x_i A_i \le B \quad \text{(预算限制)} \\
& \sum_{i \in G_k} x_i A_i \le \alpha_k B, \quad \forall k \quad \text{(信用等级比例约束)} \\
& \sum_{i=1}^N x_i A_i P_i \le R_{\max} \quad \text{(期望风险控制)} \\
& \sum_{i=1}^N x_i \le m \quad \text{(Top-$m$ 选择)} \\
& \xi_s \ge \sum_{i=1}^N x_i A_i \cdot \tilde{P}_i^{(s)} - \eta, \quad \forall s = 1,\dots,S \quad \text{(场景损失)} \\
& \eta + \frac{1}{S(1 - \beta)} \sum_{s=1}^S \xi_s \le \text{CVaR}_{\max} \quad \text{(CVaR限制)} \\
& \xi_s \ge 0, \quad \eta \in \mathbb{R}
\end{aligned}
\label{eq:cvar_model}
\end{equation}

\section{算法设计}

 
\begin{algorithm}[H]
\caption{启发式算法:贷款组合优化(含CVaR控制)}
\KwIn{
贷款数据 $\{A_i, r_i, P_i\}_{i=1}^N$;预算 $B$;Top-$m$ 限制;CVaR 上限 $\text{CVaR}_{\max}$;置信水平 $\beta$;\\
模拟场景矩阵 $\tilde{P}_i^{(s)} \in \{0,1\}^{S \times N}$
}
\KwOut{最优选择向量 $x^* \in \{0,1\}^N$}

$x_{\text{best}} \gets \mathbf{0}$;\tcp*{初始为空解}

根据评分 $\text{score}_i = A_i[r_i(1 - P_i) - P_i]$ 降序排列贷款\\
$total\_budget \gets 0$, $total\_selected \gets 0$, $x_{\text{current}} \gets \mathbf{0}$

\ForEach{$i$ in 排序后的贷款列表}{
  \If{$total\_budget + A_i > B$ 或 $total\_selected + 1 > m$}{
    continue\;
  }
  $x_{\text{current}}[i] \gets 1$\;
  $total\_budget \gets total\_budget + A_i$\;
  $total\_selected \gets total\_selected + 1$\;
}

\If{\texttt{Feasible}($x_{\text{current}}$)}{
  $x_{\text{best}} \gets x_{\text{current}}$\;
}

\BlankLine
\SetKwFunction{FMain}{Feasible}
\SetKwProg{Fn}{Function}{:}{}
\Fn{\FMain{$x$}}{
  \For{$s \gets 1$ \KwTo $S$}{
    $L_s \gets \sum_{i=1}^N x_i A_i \cdot \tilde{P}_i^{(s)}$\;
  }
  $\eta \gets$ $\beta$ 分位点的 $\{L_s\}$\;
  \For{$s \gets 1$ \KwTo $S$}{
    $\xi_s \gets \max(L_s - \eta, 0)$\;
  }
  $\text{CVaR}_\beta(x) \gets \eta + \frac{1}{S(1 - \beta)} \sum_{s=1}^S \xi_s$\;
  \Return 是否满足 $\text{CVaR}_\beta(x) \le \text{CVaR}_{\max}$ 且满足其他约束\;
}

\Return $x_{\text{best}}$\;
\end{algorithm}

\section{案例研究}
\label{sec:case_study}
\subsection{数据集描述}

\label{subsec:dataset_description}

本研究使用的数据集来自 Lending Club 平台,原始数据由 Kaggle 网站\url{https://www.kaggle.com/datasets/wordsforthewise/lending-club}公开提供。Lending Club 是美国最大的网络借贷平台之一,提供了详尽的个人借款申请及其还款情况的数据,广泛应用于学术界和工业界进行信贷风险评估、违约预测及投资组合优化等研究。

该数据集包含了从 2007 年至 2018 年的借款记录,共计数百万条样本。每条记录对应一笔贷款申请,涵盖了包括贷款金额、利率、借款人信用等级、债务收入比、贷款期限、还款状态、就业年限、收入、地址状态、房屋所有权、FICO 评分区间等在内的多维度信息。

在本研究中,我们主要筛选并保留以下变量用于建模分析:

\begin{itemize}
  \item \textbf{loan\_amnt}: 借款人申请的贷款金额,作为 $A_i$;
  \item \textbf{int\_rate}: 借款合同中约定的年利率,用于计算收益率 $r_i$;
  \item \textbf{grade}: 借款人的信用等级(A 至 G),用于分组限制;
  \item \textbf{loan\_status}: 实际贷款的还款状态(如 Fully Paid、Charged Off),用于推断违约情况;
  \item \textbf{annual\_inc}: 借款人年收入;
  \item \textbf{dti}: 债务收入比,用于辅助风险刻画;
  \item \textbf{term}: 贷款期限(如 36 months 或 60 months);
  \item \textbf{emp\_length}: 借款人工作年限;
  \item \textbf{addr\_state}: 借款人所在州;
  \item \textbf{fico\_range\_high, fico\_range\_low}: 借款人 FICO 信用评分区间;
\end{itemize}

为了满足模型中对违约概率 $P_i$ 的需求,我们将 ``loan\_status'' 字段中状态为 ``Charged Off'' 的贷款视为违约样本,其余如 ``Fully Paid''、``Current'' 等状态作为非违约样本,并基于历史频率法估算每一类贷款的违约概率。

此外,为模拟贷款违约的风险场景,我们以借款人的信用等级、FICO评分和历史违约频率为依据,构建了 $S$ 个 Monte Carlo 风险场景,用于后续 CVaR 优化模型的风险评估。

通过上述数据处理步骤,最终形成了一个结构规范、信息完备、适用于组合优化问题的数据集,为后续实证分析与建模提供了坚实的数据基础。
\subsection{使用机器学习预测违约概率 $P_i$}
\label{subsec:predict_p_i}

尽管本研究使用的数据集中所有贷款均已获得实际资助,但在资金有限、需进行优选配置的情境下,我们仍需要对这些已发放贷款的还款风险进行再评估。为此,我们引入机器学习方法,对每笔贷款的未来违约概率 $P_i$ 进行预测建模,以作为后续优化模型中的风险输入参数。

\paragraph{建模目标} 
预测函数的目标是:对于每笔已发放贷款 $i$,根据其已知特征向量 $\mathbf{x}_i$,估计其在未来发生违约的概率 $P_i = \mathbb{P}(y_i = 1 \mid \mathbf{x}_i)$。其中,$y_i=1$ 表示贷款最终发生违约(如状态为 \texttt{Charged Off}),$y_i=0$ 表示贷款最终还清(如状态为 \texttt{Fully Paid})。

\paragraph{特征构造}
我们基于借款人基本属性、贷款合同信息以及信用评级等信息构建预测特征集,涵盖但不限于以下变量:

\begin{itemize}
  \item \textbf{loan\_amnt}:贷款金额;
  \item \textbf{term}:贷款期限;
  \item \textbf{int\_rate}:贷款利率;
  \item \textbf{grade} 和 \textbf{sub\_grade}:Lending Club 信用评级;
  \item \textbf{emp\_length}:工作年限;
  \item \textbf{home\_ownership}:住房类型;
  \item \textbf{annual\_inc}:年收入;
  \item \textbf{dti}:债务收入比;
  \item \textbf{purpose}:贷款用途;
  \item \textbf{fico\_range\_high / low}:信用评分;
\end{itemize}

\paragraph{建模方法}
考虑到目标变量仍是二元状态(违约 / 未违约),我们采用监督学习的二分类方法进行建模。尝试的模型包括逻辑回归(Logistic Regression)、随机森林(Random Forest)、梯度提升树(GBDT)、极端梯度提升(XGBoost)等。

由于样本中违约样本占比相对较小,我们在训练过程中采用类别加权、欠采样等方式处理类别不平衡问题。

\paragraph{训练与评估}

我们将全部已发放贷款随机划分为训练集(70\%)与测试集(30\%),使用交叉验证调优参数,并基于测试集评估模型表现。评价指标包括准确率(Accuracy)、AUC 值(Area Under the ROC Curve)、F1 分数等。

最终,我们选择 AUC 表现最优的模型用于对所有贷款样本生成违约概率预测值 $\hat{P}_i$,作为后续优化模型中的输入。

\paragraph{说明}
虽然原始数据中每笔贷款都已实际放款,但我们的建模任务是为现实中的“再选择”提供依据。即在预算受限、资源不足时,如何在这些真实已放款的贷款中优先挑选违约概率低、预期收益高的子集,构建一个更稳健的投资组合。因此,$\hat{P}_i$ 的预测并非用于决定放款与否,而是作为组合优化的“风险估计量”,用于构建期望收益与 CVaR 等风险指标。
\subsection{使用算法求解原始模型、VaR模型与CVaR模型}
\label{subsec:model_solving}

在本节中,我们分别构建并求解三类贷款筛选优化模型:原始模型、VaR(Value at Risk)模型和 CVaR(Conditional Value at Risk)模型。三者均以最大化投资收益为目标,并在此基础上逐步引入风险控制手段,以模拟实际投资场景中对风险的不同管控需求。

\vspace{1em}
\paragraph{原始模型(期望损失约束)}
原始模型以最大化期望净收益为目标,同时设置多个线性约束,确保预算控制、信用等级平衡与风险控制等现实要求。
该模型为典型的 0-1 整数规划问题,可使用 Gurobi 等商业求解器在中小规模下获得最优解。

\vspace{1em}
\paragraph{VaR 模型(引入分位损失约束)}

VaR 模型进一步考虑极端情境下的最大可能损失,将分位数损失 $\text{VaR}_\beta$ 作为新的风险控制手段。通过构造 Monte Carlo 风险场景,我们对贷款可能发生的违约路径进行模拟,并引入辅助变量 $z_s \in \{0,1\}$ 表示某个场景是否超过 VaR 阈值 $\eta$。

由于 VaR 对应的非凸约束结构和0-1变量使得模型求解复杂度较高,通常只适用于中小规模问题,且求解结果可能存在不连续性。

\vspace{1em}
\paragraph{CVaR 模型(引入条件期望损失约束)}

为克服 VaR 模型不可微、不可凸等缺点,CVaR 模型通过引入辅助变量 $\xi_s \ge 0$ 和 VaR 估计值 $\eta$,以线性结构实现对尾部风险的精确控制。其约束具有良好的可解性与可扩展性,广泛应用于实际金融优化问题中。

该模型可转化为混合整数线性规划(MILP)形式,使用 Gurobi 等求解器可以在合理时间内获得精确解,特别适合场景规模较大的优化问题。

\vspace{1em}
\paragraph{求解策略}

在三类模型中,原始模型为最简形式,适用于风险可控或追求高收益的情境;VaR 模型在表达风险容忍边界方面具有直观优势;而 CVaR 模型则提供了更稳健的风险控制能力和更优的优化特性。

考虑到实际求解效率与解的稳定性,本研究在实现中主要采用 CVaR 模型进行主模型求解,并通过 Gurobi 求解器实现精确建模与最优解求取。对于大规模场景(如 $S > 1000$),可结合启发式算法进行初解生成与变量预选,从而提升整体求解效率。

\subsection{结果分析}
\label{subsec:result_analysis}

\section{结论}
\newpage
\bibliographystyle{unsrtnat} 
\bibliography{ref}

\end{document}
